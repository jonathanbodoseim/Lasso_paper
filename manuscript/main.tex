%% %%%%%%%%%%%%%%%%%%%%%%%%%%%%%%%%%%%%%%%%%%%%%%%%%
%% Template for working papers
%% Tommaso Di Francesco
%% University of Amsterdam
%% %%%%%%%%%%%%%%%%%%%%%%%%%%%%%%%%%%%%%%%%%%%%%%%%%
%% Version 1.0 // February 2024
%% 
\documentclass[11pt]{article}
\usepackage{style}


%% ===============================================
%% Setting the line spacing (3 options: only pick one)
% \doublespacing
% \singlespacing
\onehalfspacing
%% ===============================================

\setlength{\droptitle}{-5em} %% Don't touch

% %%%%%%%%%%%%%%%%%%%%%%%%%%%%%%%%%%%%%%%%%%%%%%%%%%%%%%%%%%
% SET THE TITLE
% %%%%%%%%%%%%%%%%%%%%%%%%%%%%%%%%%%%%%%%%%%%%%%%%%%%%%%%%%%
% TITLE:
\title{Lasso Trading
}
\author{}
% DATE:
\date{\today}

% %%%%%%%%%%%%%%%%%%%%%%%%%%%%%%%%%%%%%%%%%%%%%%%%%%%%%%%%%%
% %%%%%%%%%%%%%%%%%%%%%%%%%%%%%%%%%%%%%%%%%%%%%%%%%%%%%%%%%%
\begin{document}
% %%%%%%%%%%%%%%%%%%%%%%%%%%%%%%%%%%%%%%%%%%%%%%%%%%%%%%%%%%
% %%%%%%%%%%%%%%%%%%%%%%%%%%%%%%%%%%%%%%%%%%%%%%%%%%%%%%%%%%
% ABSTRACT
% %%%%%%%%%%%%%%%%%%%%%%%%%%%%%%%%%%%%%%%%%%%%%%%%%%%%%%%%%%
% %%%%%%%%%%%%%%%%%%%%%%%%%%%%%%%%%%%%%%%%%%%%%%%%%%%%%%%%%%
{\setstretch{.8}
\maketitle
\centering
% %%%%%%%%%%%%%%%%%%


}


% --------------------
\section{Draft}
% --------------------

Consider a simple asset pricing model in which the equilibrium price of a stochastic asset evolves accrodingly to the following law of Motion
\begin{equation}
    p_t = p^{e}_{t+1} + u_t,
\end{equation}

where $p_t$ is the equilibrium price at time $t$, $p^{e}_{t+1}$ is the expected price at time $t+1$ formed at time $t$ \textit{before} the realization of the equilibrium price and $u_t \sim N(0, \sigma^2_u)$.
We assume that agents are boundedly rational and are not aware of the true data generating process.
Instead they rely on econometric techniques to form their expectations.
Specifically we assume that agents have a linear and stationary perceived law of motion (PLM) for the asset return 
\begin{equation}
r_{t+1} = \beta \theta_t + \varepsilon_t,
\end{equation}
where $\theta_t$ is a stochastic signal that agents observe and $\varepsilon_t \sim N(0, \sigma^2_\varepsilon)$.

In order to estimate the parameter agent use Recursive Ordinary Least square of the form 

\begin{equation}
\hat{\beta_t} = \frac{\sum^{t-1}_{s=0} r_{s} \theta_{s-1}}{\sum^{t-1}_{s=0}(\theta_{s-1})^2},
\end{equation}

in which the time indexes are consistent with the timeline of events which implies that at time $t$ the last observable return that can be used for estimation is $r_{t-1}$.
The Actual Law of Motion (ALM) for the asset price obtains by substituting the PLM into the equilibrium condition

\begin{equation}
p_t = p^{e}_{t+1} + u_t = (1 + \hat{\beta_t} \theta_t) p^e_{t},
\end{equation}

where again $p^e_{t}$ stresses that the current price is not available to agents when making their forecats.
Consisten with their model

\begin{equation}
p^e_t = (1 +\hat{\beta_t} \theta_{t+1}) p_{t-1},
\end{equation}
where the use of $\hat{\beta_t}$ is justified by the belief that there is a constant homogenous $\beta$ and so it make sense for agents to use the 'smoothed' estimate obtained by considering also the return at time $t-1$ in place of $\hat{\beta_{t-1}}$.
This implies an ALM for the current price

\begin{equation}
p_t = (1 + \hat{\beta_t} \theta_t)(1 + \hat{\beta_t} \theta_{t-1}) p_{t-1} + u_t,
\end{equation}

and it follows that future returns evolve according to 

\begin{equation}
r_{t+1} = \hat{\beta_{t+1}} \theta_{t+1} + \hat{\beta_{t+1}} \theta_t + (\hat{\beta_{t+1}})^2 \theta_{t+1} \theta_t + \frac{u_t}{p_{t-1}}.
\end{equation}

\subsection{Existence of self-fulfilling equilibria}

A self-fulfilling equilibrium requires consistency between the regression coefficient that agents would estimate from the ALM and the actual $\beta$ implied by their forecast. 
Formally, let the regression coefficient that would be estimated from the ALM if agents used a constant $\beta$ be denoted by $\hat{\beta}$, we require

\begin{equation}
\hat{\beta_t} =  \beta(\hat{\beta_t}) = \frac{\operatorname{Cov}(r_{t+1},\theta_t)}{\operatorname{Var}(\theta_t)}
\end{equation}

where the notation stresses that the population parameter $\beta$ is a function of the estimated parameter $\hat{\beta_t}$.
A self-fulfilling equilibrium is a fixed point of this function, i.e. a solution to the equation $\hat{\beta} = \beta(\hat{\beta})$.

Now
\begin{equation}
\operatorname{Cov}\big(r_{t+1},\theta_t\big)
= \hat{\beta}\,\operatorname{Cov}(\theta_{t+1},\theta_t)
+ \hat{\beta}\,\operatorname{Var}(\theta_t)
+ \hat{\beta}^2\,\operatorname{Cov}(\theta_{t+1}\theta_t,\theta_t)
+ \operatorname{Cov}\!\Big(\frac{u_t}{p_{t-1}},\theta_t\Big).
\end{equation}

Given $u_t$ is mean zero and independent of $\theta$, the last term is zero, moreover if we assume that $\theta_t$ is iid
then every term except the variance is zero and we obtain
\begin{equation}
\hat{\beta}=\hat{\beta},
\end{equation}

which implies that every $\hat{\beta}$ leads to a self-fulfilling equilibrium.

\subsection{Multiple regressors}
We now move to the case in wich instead of a single regressor $\theta_t$ agents observe a vector of $K$ different signals $\theta_{t} = (\theta_{1,t}, \theta_{2,t}, \ldots, \theta_{K,t})$.
To be consistent with dimensions we write the PLM as
\begin{equation}
r_{t+1} =  \theta_t \beta + \varepsilon_t,
\end{equation}
where $\beta = (\beta_1, \beta_2, \ldots, \beta_K)^\top$ is a $K$-dimensional column vector of parameters.
The ALM follows accrodingly as
\begin{equation}
    r_{t+1} = \theta_{t+1} \hat{\beta}+ \theta_t \hat{\beta}+ (\theta_{t+1} \beta)(\theta_t \beta) + \frac{u_t}{p_{t-1}}.
\end{equation}

Defining 
\begin{equation}
    \Theta_t = \begin{pmatrix}
    \theta_{1} \\
     \theta_{2}\\
        \vdots \\
    \theta_{t} 
\end{pmatrix}
\quad 
R_t = \begin{pmatrix}
    r_{2} \\
     r_{3}\\
        \vdots \\
    r_{t+1}
\end{pmatrix},
\end{equation}
as the $t \times K$ matrix of regressors and the $t$-dimensional vector of returns up to time $t+1$,  the consistency condition requires

\begin{equation}
    \hat{\beta} = (\Theta_t^\top \Theta_t)^{-1} \Theta_t^\top R_{t} = \beta,
\end{equation}

and the ALM in matrix form is
\begin{equation}
    R_t = \Theta_{t+1} \hat{\beta}+ \Theta_t \hat{\beta}+ (\Theta_{t+1} \hat{\beta})(\Theta_t \hat{\beta}) + \frac{U_t}{P_{t-1}},
\end{equation}

where again we use the notation $U_t$ and $P_{t-1}$ to denote the vectors of shocks and prices up to the corresponding time periods.

Now asymptotically we can show that 

\begin{equation}
    \mathbb{E}[\Theta_t^\top R_t] = \mathbb{E}[\Theta_t^\top \Theta_{t+1}] \hat{\beta}+ \mathbb{E}[\Theta_t^\top \Theta_t] \hat{\beta} + \mathbb{E}[\Theta_t^\top (\Theta_{t+1} \hat{\beta})(\Theta_t \hat{\beta})] + \mathbb{E}\Big[\Theta_t^\top \frac{U_t}{P_{t-1}}\Big] = \mathbb{E}[\Theta_t^\top \Theta_t] \hat{\beta},
\end{equation}


which again implies that every $\hat{\beta}$ is a self-fullfilling equilibriums since
\begin{equation}
    \mathbb{E}[\hat{\beta}] = \mathbb{E}[(\Theta_t^\top \Theta_t)^{-1}] \mathbb{E}[\Theta_t^\top R_{t}] = \hat{\beta}.
\end{equation}

\end{document}